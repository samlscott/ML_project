\documentclass[11pt,preprint, authoryear]{elsarticle}

\usepackage{lmodern}
%%%% My spacing
\usepackage{setspace}
\setstretch{1.2}
\DeclareMathSizes{12}{14}{10}{10}

% Wrap around which gives all figures included the [H] command, or places it "here". This can be tedious to code in Rmarkdown.
\usepackage{float}
\let\origfigure\figure
\let\endorigfigure\endfigure
\renewenvironment{figure}[1][2] {
    \expandafter\origfigure\expandafter[H]
} {
    \endorigfigure
}

\let\origtable\table
\let\endorigtable\endtable
\renewenvironment{table}[1][2] {
    \expandafter\origtable\expandafter[H]
} {
    \endorigtable
}


\usepackage{ifxetex,ifluatex}
\usepackage{fixltx2e} % provides \textsubscript
\ifnum 0\ifxetex 1\fi\ifluatex 1\fi=0 % if pdftex
  \usepackage[T1]{fontenc}
  \usepackage[utf8]{inputenc}
\else % if luatex or xelatex
  \ifxetex
    \usepackage{mathspec}
    \usepackage{xltxtra,xunicode}
  \else
    \usepackage{fontspec}
  \fi
  \defaultfontfeatures{Mapping=tex-text,Scale=MatchLowercase}
  \newcommand{\euro}{€}
\fi

\usepackage{amssymb, amsmath, amsthm, amsfonts}

\def\bibsection{\section*{References}} %%% Make "References" appear before bibliography


\usepackage[round]{natbib}

\usepackage{longtable}
\usepackage[margin=2.3cm,bottom=2cm,top=2.5cm, includefoot]{geometry}
\usepackage{fancyhdr}
\usepackage[bottom, hang, flushmargin]{footmisc}
\usepackage{graphicx}
\numberwithin{equation}{section}
\numberwithin{figure}{section}
\numberwithin{table}{section}
\setlength{\parindent}{0cm}
\setlength{\parskip}{1.3ex plus 0.5ex minus 0.3ex}
\usepackage{textcomp}
\renewcommand{\headrulewidth}{0.2pt}
\renewcommand{\footrulewidth}{0.3pt}

\usepackage{array}
\newcolumntype{x}[1]{>{\centering\arraybackslash\hspace{0pt}}p{#1}}

%%%%  Remove the "preprint submitted to" part. Don't worry about this either, it just looks better without it:
\makeatletter
\def\ps@pprintTitle{%
  \let\@oddhead\@empty
  \let\@evenhead\@empty
  \let\@oddfoot\@empty
  \let\@evenfoot\@oddfoot
}
\makeatother

 \def\tightlist{} % This allows for subbullets!

\usepackage{hyperref}
\hypersetup{breaklinks=true,
            bookmarks=true,
            colorlinks=true,
            citecolor=blue,
            urlcolor=blue,
            linkcolor=blue,
            pdfborder={0 0 0}}


% The following packages allow huxtable to work:
\usepackage{siunitx}
\usepackage{multirow}
\usepackage{hhline}
\usepackage{calc}
\usepackage{tabularx}
\usepackage{booktabs}
\usepackage{caption}


\newenvironment{columns}[1][]{}{}

\newenvironment{column}[1]{\begin{minipage}{#1}\ignorespaces}{%
\end{minipage}
\ifhmode\unskip\fi
\aftergroup\useignorespacesandallpars}

\def\useignorespacesandallpars#1\ignorespaces\fi{%
#1\fi\ignorespacesandallpars}

\makeatletter
\def\ignorespacesandallpars{%
  \@ifnextchar\par
    {\expandafter\ignorespacesandallpars\@gobble}%
    {}%
}
\makeatother

\newlength{\cslhangindent}
\setlength{\cslhangindent}{1.5em}
\newenvironment{CSLReferences}%
  {\setlength{\parindent}{0pt}%
  \everypar{\setlength{\hangindent}{\cslhangindent}}\ignorespaces}%
  {\par}


\urlstyle{same}  % don't use monospace font for urls
\setlength{\parindent}{0pt}
\setlength{\parskip}{6pt plus 2pt minus 1pt}
\setlength{\emergencystretch}{3em}  % prevent overfull lines
\setcounter{secnumdepth}{5}

%%% Use protect on footnotes to avoid problems with footnotes in titles
\let\rmarkdownfootnote\footnote%
\def\footnote{\protect\rmarkdownfootnote}
\IfFileExists{upquote.sty}{\usepackage{upquote}}{}

%%% Include extra packages specified by user

%%% Hard setting column skips for reports - this ensures greater consistency and control over the length settings in the document.
%% page layout
%% paragraphs
\setlength{\baselineskip}{12pt plus 0pt minus 0pt}
\setlength{\parskip}{12pt plus 0pt minus 0pt}
\setlength{\parindent}{0pt plus 0pt minus 0pt}
%% floats
\setlength{\floatsep}{12pt plus 0 pt minus 0pt}
\setlength{\textfloatsep}{20pt plus 0pt minus 0pt}
\setlength{\intextsep}{14pt plus 0pt minus 0pt}
\setlength{\dbltextfloatsep}{20pt plus 0pt minus 0pt}
\setlength{\dblfloatsep}{14pt plus 0pt minus 0pt}
%% maths
\setlength{\abovedisplayskip}{12pt plus 0pt minus 0pt}
\setlength{\belowdisplayskip}{12pt plus 0pt minus 0pt}
%% lists
\setlength{\topsep}{10pt plus 0pt minus 0pt}
\setlength{\partopsep}{3pt plus 0pt minus 0pt}
\setlength{\itemsep}{5pt plus 0pt minus 0pt}
\setlength{\labelsep}{8mm plus 0mm minus 0mm}
\setlength{\parsep}{\the\parskip}
\setlength{\listparindent}{\the\parindent}
%% verbatim
\setlength{\fboxsep}{5pt plus 0pt minus 0pt}



\begin{document}



\begin{frontmatter}  %

\title{Data Science: Machine Learning}

% Set to FALSE if wanting to remove title (for submission)




\author[Add1]{Samantha Scott}
\ead{20945043@sun.ac.za}





\address[Add1]{Stellenbosch University, Cape Town, South Africa}



\vspace{1cm}


\begin{keyword}
\footnotesize{
Machine Learning \sep Heart Disease Prediction \sep Random Forests \\
\vspace{0.3cm}
}
\end{keyword}



\vspace{0.5cm}

\end{frontmatter}


\newpage
\renewcommand{\contentsname}{Table of Contents}
{\tableofcontents}
\newpage

%________________________
% Header and Footers
%%%%%%%%%%%%%%%%%%%%%%%%%%%%%%%%%
\pagestyle{fancy}
\chead{}
\rhead{Predicting Heart Disease}
\lfoot{}
\rfoot{\footnotesize Page \thepage}
\lhead{}
%\rfoot{\footnotesize Page \thepage } % "e.g. Page 2"
\cfoot{}

%\setlength\headheight{30pt}
%%%%%%%%%%%%%%%%%%%%%%%%%%%%%%%%%
%________________________

\headsep 35pt % So that header does not go over title




\hypertarget{introduction}{%
\section{Introduction}\label{introduction}}

The following paper is a comparison between two Machine Learning
algorithms, namely Random Forests and Support Vector Machines, as
prediction tools. Using a Linear Regression model as a baseline, the
RMSE scores are compared.

\hypertarget{research-question}{%
\section{Research Question}\label{research-question}}

\hypertarget{data-and-methodology}{%
\section{Data and Methodology}\label{data-and-methodology}}

The data used in this investigation is heart disease data from Kaggle.

\hypertarget{results}{%
\section{Results}\label{results}}

\hypertarget{linear-regression}{%
\subsection{Linear Regression}\label{linear-regression}}

\begin{verbatim}
## 
## Call:
## lm(formula = heart_disease_present ~ ., data = heart_d)
## 
## Residuals:
##      Min       1Q   Median       3Q      Max 
## -0.81385 -0.23535 -0.07213  0.25418  0.90884 
## 
## Coefficients:
##                                        Estimate Std. Error t value Pr(>|t|)    
## (Intercept)                          -0.3435048  0.4431192  -0.775 0.439319    
## slope_of_peak_exercise_st_segment     0.0995764  0.0609650   1.633 0.104282    
## resting_blood_pressure                0.0008023  0.0017905   0.448 0.654683    
## chest_pain_type                       0.1120314  0.0329010   3.405 0.000828 ***
## num_major_vessels                     0.1438576  0.0328175   4.384 2.06e-05 ***
## fasting_blood_sugar_gt_120_mg_per_dl -0.0479164  0.0801823  -0.598 0.550921    
## resting_ekg_results                   0.0244777  0.0286543   0.854 0.394196    
## serum_cholesterol_mg_per_dl           0.0005870  0.0005571   1.054 0.293524    
## oldpeak_eq_st_depression              0.0616716  0.0334386   1.844 0.066907 .  
## sex                                   0.2413338  0.0638961   3.777 0.000220 ***
## age                                  -0.0028883  0.0036829  -0.784 0.434008    
## max_heart_rate_achieved              -0.0014835  0.0016266  -0.912 0.363063    
## exercise_induced_angina               0.1960958  0.0695805   2.818 0.005412 ** 
## ---
## Signif. codes:  0 '***' 0.001 '**' 0.01 '*' 0.05 '.' 0.1 ' ' 1
## 
## Residual standard error: 0.3685 on 167 degrees of freedom
## Multiple R-squared:  0.4899, Adjusted R-squared:  0.4532 
## F-statistic: 13.36 on 12 and 167 DF,  p-value: < 2.2e-16
\end{verbatim}

\begin{verbatim}
## [1] 0.3684575
\end{verbatim}

\hypertarget{random-forests}{%
\subsection{Random Forests}\label{random-forests}}

\begin{verbatim}
## [1] 0.4178554
\end{verbatim}

\begin{verbatim}
##    
##      0  1
##   0 16  8
##   1  7 23
\end{verbatim}

\begin{verbatim}
## [1] 0.7222222
\end{verbatim}

\hypertarget{support-vector-machine}{%
\subsection{Support Vector Machine}\label{support-vector-machine}}

\hypertarget{conclusion}{%
\section{Conclusion}\label{conclusion}}

\hypertarget{reference-list}{%
\section{Reference List}\label{reference-list}}

\bibliography{Tex/ref}





\end{document}
